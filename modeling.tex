     Analyzing RBAC models using a formal method, such as the B method or Event-B may assist system developers to obtain a correct and consistent specification.  By knowing RBAC constraints, formal specifications can play a significant role to examine security policies, and determine whether a particular system is secure at every point of time.
The B method is one of the abstract formal languages that uses a special language named “Abstract Machine Notation (AMN)”.  The idea is to use mathematical notations (relying on mathematical logic and set theory) to formalize specifications for the purpose of formally verifying correctness and consistency of a specification.

\subsection{Modeling the Basic RBAC Components}

To formalize the basic RBAC elements, it is essential to define the sets of USER and ROLES as well as UA and PA assignments.  In the EMS system, $USER$ set contains subsets of $students$, $guardians$ and $staff$.  The latter itself contains subsets of $teachers$ and $headmasters$.  Therefore, these subsets must be declared as variables in the AMN: 

\begin{align*}
staff, students, studentGuarian, teachers, manager
\end{align*}


Now, we can declare, within INVARIANT clause, that each of those variables is a subset of $USER$ set, and some are also subsets of $staff$ subset, provided that no element of each subset can be found into two different subsets.  For example, if a student has been registered into $students$ subset, then that student cannot be found as a member of $staff subset$ (i.e. $ students \cap staff = \{\}$).  This can be expressed using the B specification as follows:


\begin{align*}
students \subseteq USER  \wedge \\
staff \subseteq USER  \wedge \\
students \cap staff = \{ \}  \wedge \\
studentGuardian \subseteq USER   \wedge \\
studentGuardian\cap students = \{ \}  \wedge \\
manager \subseteq staff   \wedge \\
teachers \subseteq staff   \wedge \\
manager \cap teachers = \{ \} \\ 
\end{align*}



To model the User Assignment (UA), which assigns each user to a particular role, we need to define a relation between USER set and ROLES set.  This relation is, in fact, a set of elements, where each element is a pair of $user$ and $role$.  For example, if a user $u$ has been assigned to a role $r$, then there is an element for the relation can be defined as: $\{(u, r)\}$.

Since elements of the relation between USER and ROLES are subject to change (i.e. assigning new users to roles, or revoking existing assignments), we represent this relation as a variable, and we give it the name: $hasTheRole$.  The declaration of that within the INVARIANT clause would be as: 

\begin{align*}
hasTheRole \in USER \leftrightarrow ROLES
\end{align*}


To model the Permissions Assignments (PA), we do not need to define a new set for permissions, or a direct relation (as in the UA assignment) between ROLES and those permissions.  Since any permission is actually a pair of operation and object, we will be specifying (within preconditions of each operation (function)) in the system, that this operation is only performed by users who have been given a particular role.

For example, users who can add a submission are only students.  Therefore, we say, as a precondition of the operation, that \textit{“who has the role of ‘student’, can perform this operation”}.  This is expressed by the B specification as in the following algorithm:

\begin{algorithm}                      % enter the algorithm environment
\caption{Assigning "add a submission" permission to "student" role}          % give the algorithm a caption
\label{alg1}                          
%\begin{algorithmic}  


PRE \\

\quad ${\bf hasTheRole[\{s\}] = \{student\}} \wedge$\\
 
\quad $report \in SUBMISSION \wedge$ \\

\quad $report \notin reports \wedge$ \\

\quad $u \in studies [\{s\}] \wedge$ \\

\quad $card(submit[\{s\}] \cap reportFor^{-1} [\{u\}]$ \\

\quad $< maxSubmissions$
 


THEN \\ 

\quad $report \gets reports \cup \{report\} \mid\mid \\$

\quad $submits \gets submits \cup \{s \mapsto report\} \mid\mid \\$

\quad $reportFor \gets reportFpr \cup \{report \mapsto u\}$
 

END
%\end{algorithmic} 
\end{algorithm} 


\subsection{Modeling the RBAC Constraints}

     We begin with mutual exclusiveness of roles  (Separation of Duties).  As has been mentioned earlier, there are two conflicted sets of roles in the EMS system.  Since roles are previously known and defined for the system, the two conflicted sets of roles should be initialized within the INITIALIZATION clause as:

\begin{align*}
sod1 &= \{teacher, headteacher, headmaster\}, \\
sod2 &= \{student, student_guardian\}
\end{align*}

Each of $sod1$ and $sod2$ is a subset of ROLES set, and the intersection between them is equal to an empty set.  Therefore, this must be declared within INVARIANT clause:

\begin{flalign*}
sod1 \subseteq ROLES   \wedge \\
sod2 \subseteq ROLES   \wedge \\
sod1 \cap sod2 = \{ \}
\end{flalign*}

Now, we can formulate the condition of Separation of Duties (SoD) using the B specification, as follows:
\begin{align*}
\forall(u, role_1, role_2) . (u \in USER  \wedge role_1 \in sod_1 \\ \wedge role_2  \in sod_2  \wedge role_1  \in   hasTheRole[\{u\}]  \\       \Rightarrow \neg (role_2 \in hasTheRole[\{u\}] )   \\
\wedge
\end{align*}

\begin{align*}
\forall(u, role1, role2) . (u \in USER \wedge role1  \in sod1 
\\ \wedge role2  \in sod2 \wedge role2 \in   hasTheRole[\{u\}]    \\ \Rightarrow \neg (role1 \in hasTheRole[\{u\}] )  \\
\end{align*}

The above two predicates mean, respectively, that for every user \textit{u} and two given roles \textit {role1} and \textit{role2}, where $ u \in USER $ and $ role1 \in sod1 $ and $ role2 \in sod2  $ and that \textit{role1} belongs to the set of roles given to the user \textit{u}, then the \textit{role2} must not belong to the roles of the user \textit{u}.  Likewise, for every user \textit{u} and two given roles \textit{role1} and \textit{role2}, where $ u \in USER $ and $ role1 \in sod1 $ and $ role2 \in sod2  $  and that \textit{role2}  belongs to the set of roles given to the user \textit{u}, then the \textit{role1} must not belong to the roles of the user \textit{u}.

The second type of restrictions is that which is called  “Cardinality Constraints’.  In the EMS system, the number of users who will be assigned to $headmaster$ role must not be greater than 1.  Hence, the cardinality of users who have the role $headmaster$ must be less than or equal to 1.  This is expressed as: 

\begin{align*}
\forall (headmaster). (headmaster \in sod1  \Rightarrow \\
                            card(hasTheRole^{-1} [\{headmaster\}]  \leq 1 )  
\end{align*}


\subsection{Modeling the Role Hierarchy (RH)}
     Head teachers have their own permissions, such as reviewing records.  However, they still are teachers, and, therefore, they can perform operations that are given to teachers.  We need to formulate a predicate that satisfies the concept of inheritance, where head teachers can inherit permissions from $teacher$ role.  In the B specification, this is expressed as:

\begin{align*}
\forall (u, headteacher) . (headteacher  \in sod1  \wedge  \\ u \mapsto headteacher \in hasTheRole
\\ \Rightarrow headteacher  \in hasTheRole[\{u\}]  \\ \wedge  teacher  \in hasTheRole[\{u\}] )
\end{align*}

    Which means that for every user $u$ and $headteacher$ role, where $ headteacher \in sod1$  and that user has been assigned to the role $headteacher$, then the user $u$ would become have the role $teacher$ in addition to the role $headteacher$. \\
