Security is now becoming a very important aspect when it comes to the development of modern systems. Many critical systems, especially those which handle commercial transactions, enforce different security policies to their systems in order to ensure that existing data and information flow are enough secure to preserve confidentiality. If there is any defect in applying any security policy to a particular system, as if specifying ambiguous properties or defining inconsistent model, then the system might lack reliability. Hence, formal methods are widely used to verify the correctness and consistency of most of security models, including RBAC policies, as they rely on mathematical logic and set theory [3].

Role Based Access Control (RBAC) model is regarded as a successful alternative to discretionary and mandatory access control models.  Efficiently, RBAC can be more suitable for commercial systems, since it depends essentially on assigning different users to particular roles [4]. However, RBAC policy is not restricted to business systems only, but also to any critical system that needs to determine whether a given subject (user) is allowed to access a certain object (resource). 
This paper discusses a study case: Electronic Marking System (EMS) in Oman Secondary Schools, where RBAC security policy needs to be applied to.  EMS aims to provide a consolidated environment, where each member within a school has the right to access the system within the powers (authorizations) given to him/her. For example, teachers can access the system for the purpose of adding, editing or deleting marks.  Whereas, students have the authorization to submit reports and view their grades.
