The B method \cite{Abrial-BBook-1996} is a formal method for specifying, refining and implementing software.  B is a refinement methodology as it allows the system developer to start with an abstract model of the system considering its context, and gradually add details to the model leading to a sequence of more concrete models until the final implementation is reached.  The model development process creates a number of \textit{proof obligations}, which guarantee the correctness of the model as well as any desired invariants (properties) that the model of the system should preserve. Proof obligations can then be proven by automatic or interactive theorem provers or model checking tools, and the model itself can be simulated in runtime.  The proving of obligations as well as checking invariants and simulating the model are functions that are well supported by tools such as \textit{ProB} \cite{LeuschelB03}.

For the rest of the paper, we do not use the refinement part of B, and we constrain ourselves to the use of B as a formal language to specify one mode of the RBAC policies for the case of e-marking systems.   Therefore, we revisit the formalisation of the desirable properties in the system in Section \ref{sec:modeling}.
